\documentclass[12pt, a4paper]{article}

%% Change indent
\usepackage{enumitem}
\setlist[itemize]{leftmargin=*}
\setlist[enumerate]{leftmargin=*}
\setlist[description]{leftmargin=*}

%Maths
\usepackage{amsmath,amssymb,amsbsy,amscd,amstext,mathtools,amsxtra}
\setlength{\abovedisplayskip}{0pt}
\setlength{\belowdisplayskip}{0pt}
\setlength{\abovedisplayshortskip}{0pt}
\setlength{\belowdisplayshortskip}{0pt}

\newcommand{\lb}{\left (}
\newcommand{\rb}{\right )}
\newcommand{\diag}{\text{diag }}
\newcommand{\Ln}{\text{\bf Ln }}
\newcommand{\Exp}{\text{\bf Exp }}
\newcommand{\vv}{\boldsymbol{v}}
\renewcommand{\AA}{\boldsymbol{A}}
\newcommand{\BB}{\boldsymbol{B}}
\newcommand{\CC}{\boldsymbol{C}}
\newcommand{\DD}{\boldsymbol{D}}
\newcommand{\GG}{\boldsymbol{G}}
\newcommand{\AAA}{\boldsymbol{a}}
\newcommand{\BBB}{\boldsymbol{b}}
\newcommand{\CCC}{\boldsymbol{c}}
\newcommand{\DDD}{\boldsymbol{d}}
\newcommand{\GGG}{\boldsymbol{g}}
\newcommand{\KK}{\boldsymbol{K}}
\newcommand{\kk}{\boldsymbol{k}}
\newcommand{\KKb}{\bar{\KK}}
\newcommand{\Xb}{\bar{X}}
\newcommand{\Nb}{\bar{N}}
\newcommand{\kkappa}{\boldsymbol{\kappa}}
\newcommand{\mmu}{\boldsymbol{\mu}}
\newcommand{\bX}{\bar{X}}
\newcommand{\bN}{\bar{N}}
\newcommand{\bA}{\bar{A}}
\newcommand{\bvv}{\bar{\vv}}
\newcommand{\bKK}{\bar{\KK}}
\newcommand{\tKK}{\tilde{\KK}}
\newcommand{\bkkappa}{\bar{\kkappa}}
\newcommand{\bT}{\bar{T}}
\newcommand{\bTK}{{\bar{T}_K}}
\newcommand{\II}{\boldsymbol{I}}

\newcommand{\Nfb}{{\bar{N^f}}}
\newcommand{\Nrb}{{\bar{N^r}}}
\newcommand{\Nf}{{N^f}}
\newcommand{\Nr}{{N^r}}
\newcommand{\Ne}{{N^e}}
\newcommand{\Nef}{{N^{ef}}}
\newcommand{\Ner}{{N^{er}}}
\newcommand{\Ni}{{N^i}}
\newcommand{\Ae}{{A^e}}
\newcommand{\Ar}{{A^r}}

\newcommand{\tX}{\tilde{X}}
\newcommand{\dX}{{\dot{X}}}
\newcommand{\dx}{{\dot{x}}}

\newcommand{\hp}{\circ}         %Hadamard or Schur product.
\newcommand{\half}{\frac{1}{2}}  

\newcommand{\vbond}{{\Huge $\rightharpoondown$}}
\newcommand{\RT}{RT}

\newcommand{\xx}{\chi}
\newcommand{\xt}{\tilde{x}}
\newcommand{\Xt}{\tilde{X}}
\newcommand{\xxt}{\tilde{\xx}}
\newcommand{\yt}{\tilde{y}}
\newcommand{\ut}{\tilde{u}}

\renewcommand{\ss}[1]{\bar{#1}}

\newcommand{\bK}{K^M}


\newcommand{\Ass}{\bar{A}} 
\newcommand{\FF}{\boldsymbol{F}}

\newcommand{\Vt}{{\tilde{V}}}


\newcommand{\reacul}[2]{
  {\; \xrightleftharpoons[#2]{#1} \;}
}

\newcommand{\reacu}[1]{
  \reacul{#1}{}
}

\newcommand{\reac}{
  \reacu{}
}


%% Bond-Graph commands
%%\usepackage{BG}
%% Bond-Graph commands
\newcommand{\BG}[1]{\text{\sffamily\textbf{#1}}}
\newcommand{\AmpE}{\BG{AE }}
\newcommand{\AF}{\BG{AF }}
\newcommand{\C}{\BG{C }}
\newcommand{\CS}{\BG{CS }}
\newcommand{\CSW}{\BG{CSW }}
\newcommand{\CSWw}{\BG{CSW}}
\newcommand{\De}{\BG{De }}
\newcommand{\Df}{\BG{Df }}
\newcommand{\I}{\BG{I }}
\newcommand{\INTF}{\BG{INTF }}
\newcommand{\ISW}{\BG{ISW}}
\newcommand{\R}{\BG{R }}
\newcommand{\Se}{\BG{Se }}
\newcommand{\See}{\BG{Se}}
\newcommand{\Sf}{\BG{Sf }}
\newcommand{\Sff}{\BG{Sf}}
\newcommand{\Sv}{\BG{Sv}}
\renewcommand{\SS}{\BG{SS }}
\newcommand{\one}{\BG{1 }}
\newcommand{\zero}{\BG{0 }}
\newcommand{\One}{\BG{1}}
\newcommand{\Zero}{\BG{0}}
\newcommand{\zeroo}{\BG{0 }}
\newcommand{\TF}{\BG{TF }}
\newcommand{\GY}{\BG{GY }}

\renewcommand{\Re}{\BG{Re }}
\newcommand{\mRe}{\BG{mRe }}
\newcommand{\sRe}{\BG{sRe }}
\newcommand{\iRe}{\BG{iRe }}
\newcommand{\Rr}{\BG{Rr }}
\newcommand{\Rrr}{\BG{Rr2 }}

%% Bond-Graph components with label
\newcommand{\BGL}[2]{$\BG{#1}$:$\mathbf{#2}$} %Generic

\newcommand{\BC}[1]{\BGL{C}{#1}}
\newcommand{\BCS}[1]{\BGL{CS}{#1}}
\newcommand{\BI}[1]{\BGL{I}{#1}}
\newcommand{\BR}[1]{\BGL{R}{#1}}

\newcommand{\BSS}[1]{\BGL{SS}{#1}}
\newcommand{\BSe}[1]{\BGL{Se}{#1}}
\newcommand{\BSf}[1]{\BGL{Sf}{#1}}
\newcommand{\BSv}[1]{\BGL{Sv}{#1}}
\newcommand{\BDe}[1]{\BGL{De}{#1}}
\newcommand{\BDf}[1]{\BGL{Df}{#1}}

\newcommand{\BTF}[1]{\BGL{TF}{#1}}
\newcommand{\BGLY}[1]{\BGL{GY}{#1}}

\newcommand{\BRe}[1]{\BGL{Re}{#1}}
\newcommand{\BmRe}[1]{\BGL{mRe}{#1}}
\newcommand{\BsRe}[1]{\BGL{sRe}{#1}}
\newcommand{\BiRe}[1]{\BGL{iRe}{#1}}
% Figs
\usepackage{graphicx,subfigure,fancybox,color}

\newcommand{\Fig}[2]{
 \includegraphics[width=#2\linewidth]{Figs/#1.pdf}
}

\newcommand{\RotFig}[2]{
 \includegraphics[width=#2\linewidth,angle=90,origin=c]{Figs/#1.pdf}
}

\newcommand{\SubFig}[3]{
 \subfigure[#2]{
   \includegraphics[width=#3\linewidth]{Figs/#1.pdf}
   \label{subfig:#1}
 }
}

\newcommand{\SubPic}[3]{
 \subfigure[#2]{
   \resizebox{#3\linewidth}{!}{\input{Pics/#1}}
   \label{subfig:#1}
 }
}

\newcommand{\SubFreq}[2]{
  \SubFig{ABCA_lin_bode_#1#2}{$\GG_{#1#2}(\omega)$}{0.3}
}

\newcommand{\SubStep}[2]{
  \SubFig{ABCA_lin_step_#1#2}{$\GGG_{#1#2}(t)$}{0.3}
}

\newcommand{\Pic}[2]{
  \resizebox{#2\linewidth}{!}{\input{Pics/#1}}
 }

\newcommand{\SubModel}[2]{
\begin{figure}[htbp]
  \centering
  \Fig{#1_abg}{#2}
  \caption{Submodel: \Bg{#1}}
  \label{fig:#1}
\end{figure}
}

% %% XYpic diagrams
% \usepackage[all,pdftex]{xy}
% %%\usepackage[all]{xy}
% \usepackage{array}
% \usepackage{color}
% \input{BG_pic}

%% Irreversible reaction
\newcommand{\ire}[2]{
\ar@^{>}@<0.3ex>[#1]^{#2}
}

%% Bond-Graph commands
\newcommand{\Bg}[1]{\text{\sffamily\textbf{#1}}}

%% Links
% \usepackage[pagebackref]{hyperref}
% \hypersetup{colorlinks=true,citecolor=blue,linkcolor=blue,urlcolor=blue}
%\usepackage{doi}

%%\usepackage{endfloat}

\begin{document}
\title{Hierarchical Bond Graph Modelling of Biochemical Networks:
  Figures and auto-generated equations.}

\maketitle

\section{Auto-generated equations.}

%file: lamkus02_{dae}.tex
%differential-algebraic equations
\begin{equation}
\begin{aligned}
\dot x_{1} &=
{
2 u_{1} - u_{10} - u_{11} + u_{16} + u_{17} + u_{2}
}
\cr
\dot x_{2} &=
{
 - u_{1}
}
\cr
\dot x_{3} &=
{
 - u_{1} + u_{10} + u_{11} - u_{16} - u_{17} - u_{2}
}
\cr
\dot x_{4} &=
{
 - u_{2}
}
\cr
\dot x_{5} &=
{
 - u_{12} - u_{13} + u_{17} - u_{8}
}
\cr
\dot x_{6} &=
{
u_{2}
}
\cr
\dot x_{7} &=
{
u_{3} + u_{4}
}
\cr
\dot x_{8} &=
{
u_{16} - u_{3}
}
\cr
\dot x_{9} &=
{
u_{3} - u_{4} - u_{8}
}
\cr
\dot x_{10} &=
{
 - u_{10} + u_{8}
}
\cr
\dot x_{11} &=
{
 - u_{6} + u_{9}
}
\cr
\dot x_{12} &=
{
u_{6}
}
\cr
\dot x_{13} &=
{
 - u_{8} + u_{9}
}
\cr
\dot x_{14} &=
{
u_{8} - u_{9}
}
\cr
\dot x_{15} &=
{
 - u_{5} + u_{7}
}
\cr
\dot x_{16} &=
{
u_{10} - u_{7}
}
\cr
\dot x_{17} &=
{
 - u_{11} + u_{5}
}
\cr
\dot x_{18} &=
{
u_{11} - u_{9}
}
\cr
\dot x_{19} &=
{
u_{14} - u_{16}
}
\cr
\dot x_{20} &=
{
u_{12} + u_{13} - u_{15}
}
\cr
\dot x_{21} &=
{
 - u_{14} + u_{15}
}
\cr
\dot x_{22} &=
{
0
}
\end{aligned}
\end{equation}
\begin{equation}
\begin{aligned}
y_{1} &=
{
{\left (v_{adk} {\left ( - k_{adp}^2 x_{1}^2 + k_{amp} k_{atp} x_{2} x_{3}\right )}\right )} \over {\left (k_{adk} + k_{adp}^2 x_{1}
^2 \rho_{adk} - k_{amp} k_{atp} x_{2} x_{3} \rho_{adk} + k_{amp} k_{atp} x_{2} x_{3}\right )}
}
\cr
y_{2} &=
{
{\left (v_{ck} {\left ( - k_{adp} k_{pcr} x_{1} x_{6} + k_{atp} k_{cr} x_{3} x_{4}\right )}\right )} \over {\left (k_{adp} k_{pcr} x_{1}
x_{6} \rho_{ck} - k_{atp} k_{cr} x_{3} x_{4} \rho_{ck} + k_{atp} k_{cr} x_{3} x_{4} + k_{ck}\right )}
}
\cr
y_{3} &=
{
{\left (v_{ald} {\left ( - k_{dhap} k_{gap} x_{7} x_{9} + k_{fbp} x_{8}\right )}\right )} \over {\left (k_{ald} + k_{dhap} k_{gap} x_{7}
x_{9} \rho_{ald} - k_{fbp} x_{8} \rho_{ald} + k_{fbp} x_{8}\right )}
}
\cr
y_{4} &=
{
{\left (v_{t\pi} {\left ( - k_{dhap} x_{7} + k_{gap} x_{9}\right )}\right )} \over {\left (k_{dhap} x_{7} \rho_{t\pi} - k_{gap} x_{9}
\rho_{t\pi} + k_{gap} x_{9} + k_{t\pi}\right )}
}
\cr
y_{5} &=
{
{\left (v_{eno} {\left (k_{p2g} x_{15} - k_{pep} x_{17}\right )}\right )} \over {\left (k_{eno} - k_{p2g} x_{15} \rho_{eno} + k_{p2g}
x_{15} + k_{pep} x_{17} \rho_{eno}\right )}
}
\cr
y_{6} &=
{
{\left (v_{fout} {\left (k_{lac} x_{11} - k_{laco} x_{12}\right )}\right )} \over {\left (k_{fout} - k_{lac} x_{11} \rho_{fout} + k_{lac}
x_{11} + k_{laco} x_{12} \rho_{fout}\right )}
}
\cr
y_{7} &=
{
{\left (v_{pgm} {\left ( - k_{p2g} x_{15} + k_{p3g} x_{16}\right )}\right )} \over {\left (k_{p2g} x_{15} \rho_{pgm} - k_{p3g} x_{16}
\rho_{pgm} + k_{p3g} x_{16} + k_{pgm}\right )}
}
\cr
y_{8} &=
{
{\left (v_{gapdh} {\left ( - k_{dpg} k_{nadh} x_{10} x_{14} + k_{gap} k_{nad} k_{p} x_{13} x_{5} x_{9}\right )}\right )} \over {\left (
k_{dpg} k_{nadh} x_{10} x_{14} \rho_{gapdh} - k_{gap} k_{nad} k_{p} x_{13} x_{5} x_{9}
\rho_{gapdh} + k_{gap} k_{nad} k_{p} x_{13} x_{5} x_{9} + k_{gapdh}\right )}
}
\cr
y_{9} &=
{
{\left (v_{ldh} {\left ( - k_{lac} k_{nad} x_{11} x_{13} + k_{nadh} k_{pyr} x_{14} x_{18}\right )}\right )} \over {\left (k_{lac} k_{nad}
 x_{11} x_{13} \rho_{ldh} + k_{ldh} - k_{nadh} k_{pyr} x_{14} x_{18} \rho_{ldh} + k_{nadh}
k_{pyr} x_{14} x_{18}\right )}
}
\cr
y_{10} &=
{
{\left (v_{pgk} {\left ( - k_{adp} k_{dpg} x_{1} x_{10} + k_{atp} k_{p3g} x_{16} x_{3}\right )}\right )} \over {\left (k_{adp} k_{dpg}
x_{1} x_{10} \rho_{pgk} - k_{adp} k_{dpg} x_{1} x_{10} - k_{atp} k_{p3g} x_{16} x_{3}
\rho_{pgk} - k_{pgk}\right )}
}
\cr
y_{11} &=
{
{\left (v_{pk} {\left ( - k_{adp} k_{pep} x_{1} x_{17} + k_{atp} k_{pyr} x_{18} x_{3}\right )}\right )} \over {\left (k_{adp} k_{pep}
x_{1} x_{17} \rho_{pk} - k_{adp} k_{pep} x_{1} x_{17} - k_{atp} k_{pyr} x_{18} x_{3} \rho_{pk}
 - k_{pk}\right )}
}
\cr
y_{12} &=
{
{\left (k_{gly} x_{22} v_{gpa} {\left ( - k_{g1p} x_{20} + k_{p} x_{5}\right )}\right )} \over {\left (k_{g1p} k_{gly} x_{20} x_{22}
\rho_{gpa} - k_{gly} k_{p} x_{22} x_{5} \rho_{gpa} + k_{gly} k_{p} x_{22} x_{5} + k_{gpa}\right )}
}
\cr
y_{13} &=
{
{\left (k_{gly} x_{22} v_{gpb} {\left ( - k_{g1p} x_{20} + k_{p} x_{5}\right )}\right )} \over {\left (k_{g1p} k_{gly} x_{20} x_{22}
\rho_{gpb} - k_{gly} k_{p} x_{22} x_{5} \rho_{gpb} + k_{gly} k_{p} x_{22} x_{5} + k_{gpb}\right )}
}
\cr
y_{14} &=
{
{\left (v_{pgi} {\left ( - k_{f6p} x_{19} + k_{g6p} x_{21}\right )}\right )} \over {\left (k_{f6p} x_{19} \rho_{pgi} - k_{g6p} x_{21}
\rho_{pgi} + k_{g6p} x_{21} + k_{pgi}\right )}
}
\cr
y_{15} &=
{
{\left (v_{pglm} {\left ( - k_{g1p} x_{20} + k_{g6p} x_{21}\right )}\right )} \over {\left (k_{g1p} x_{20} \rho_{pglm} - k_{g1p} x_{20}
- k_{g6p} x_{21} \rho_{pglm} - k_{pglm}\right )}
}
\cr
y_{16} &=
{
{\left (v_{pfk} {\left ( - k_{adp} k_{fbp} x_{1} x_{8} + k_{atp} k_{f6p} x_{19} x_{3}\right )}\right )} \over {\left (k_{adp} k_{fbp}
x_{1} x_{8} \rho_{pfk} - k_{atp} k_{f6p} x_{19} x_{3} \rho_{pfk} + k_{atp} k_{f6p} x_{19}
x_{3} + k_{pfk}\right )}
}
\cr
y_{17} &=
{
{\left (v_{atpase} {\left ( - k_{adp} k_{p} x_{1} x_{5} + k_{atp} x_{3}\right )}\right )} \over {\left (k_{adp} k_{p} x_{1} x_{5}
\rho_{atpase} - k_{atp} x_{3} \rho_{atpase} + k_{atp} x_{3} + k_{atpase}\right )}
}
\end{aligned}
\end{equation}


\section{Figures}

\begin{figure}[htbp]
  \centering 
  \SubFig{ABCAo_abg}{Example: external efforts  $\mu_F$ and $\mu_R$.}{0.45}
  \SubFig{ABCAv_abg}{Example: external flow $v_e$}{0.45}
  \SubFig{Open_abg}{General}{0.9}
  \caption{Closed \& open systems. (a) \& (b) Simple examples corresponding
    to reaction schemes () and ()  with
    imposed concentrations and flows, respectively.
    % 
    The \Bg{C:A}, \Bg{C:B}, and \Bg{C:C} components accumulate the
    species $A$, $B$ and $C$; and the \Bg{Re:r1}, \Bg{Re:r2} and
    \Bg{Re:r3} represent reactions 1, 2 and 3. The \Bg{SS:[F]},
    \Bg{SS:[R]} and \Bg{SS:[v\_e]}
    components are the energy ports converting a closed system to an open one.
    % 
    The system inside the dashed box is a closed system; when
    \Bg{SS} components (representing energy ports) are added the
    system becomes open. (c) General case. The
    bond symbols
    $\rightharpoondown$ correspond to \emph{vectors} of bonds; 
    $\mathcal{C}$, $\mathcal{R}e$, $\mathcal{SS}$, \zero and \one
    symbols correspond to arrays of the associated components; and 
    $\mathcal{TF}$ components represent the intervening
    junction structure which transmits energy, comprising bonds, junctions
    and \TF components. $\mathcal{SS}:[\mu_e]$ represents a vector of
    energy ports imposing external chemical potentials;
    $\mathcal{SS}:[v_E]$ represents a vector of energy ports imposing
    external flows. In particular, with respect to (a) \& (b),
    % 
    $\mathcal{C}$ corresponds to the species represented by \Bg{C:A},
    \Bg{C:B}, and \Bg{C:C},
    % 
    $\mathcal{R}e$ corresponds to the reactions represented by \Bg{Re:r1}, \Bg{Re:r2} and
    \Bg{Re:r3} and
    % 
    $\mathcal{TF}:N^f_i$ and $\mathcal{TF}:N^r_i$ summarise the
    connections between reactions and species. 
    % 
    With respect to (a), $\mathcal{SS}:[\mu_e]$ corresponds to \Bg{SS:[F]},
    \Bg{SS:[R]}; with respect to (b), $\mathcal{SS}:[v_E]$ corresponds
    to \Bg{SS:[v\_e]}. 
%
    $\mathcal{TF}:N^f_e$ and $\mathcal{TF}:N^r_e$ summarise the
    connections between reactions and ports, and $\mathcal{TF}:N_E$
    between species and ports, in (a) \& (b).  }
 \label{fig:closed-open-systems}
\end{figure}

\begin{figure}[htbp]
  \centering
  \SubFig{ABCA_sim_v}{Closed system: $V$}{0.45}
  \SubFig{ABCAo_sim_v}{Open system: $V$}{0.45}
  \SubFig{ABCA_sim_x}{Closed system: $X$}{0.45}
  \SubFig{ABCAo_sim_x}{Open system: $X$}{0.45}
  \caption{Simulation of closed \& open systems. The parameters are:
    $K_a=K_b=1$, $K_c=K_f =2$, $K_r=0.25$, $X_f=X_r=1$, $\kappa_1=1$,
    $\kappa_2=2$ \& $\kappa_3=3$.
    % 
    (a) As this is a
    closed system, the three reaction flows $v_1$--$v_3$ become zero
    as time increases.
    % 
    (b) As this is an open system, the three reaction flows
    $v_1$--$v_3$ do not become zero as the system has an external
    driver. However, as the states become constant, the three flows
    become equal.  
    % 
    (c) Although the closed system flows become zero the states do not, 
    as the system has a conserved moiety: the three states have a 
    constant sum $x_a+x_b+x_c=4$.
    % 
    (d) The three states tend to different values to those in (c); but
    the moiety is still conserved.  }
  \label{fig:sim_ABCAo}
\end{figure}

\begin{figure}[htbp]
  \centering
  \SubFig{ABCA_sim_P}{Closed system}{0.45}
  \SubFig{ABCAo_sim_P}{Open system}{0.45}
  \caption{Energy flows for the system illustrated in Figure () and simulated in Figure ():
    $P_E$ is the external energy flowing into the system, 
    $P_C$ is the energy flowing into the three \C
    components, and $P_R$ is the energy flowing into (and dissipated by)
    the three \Re components.
    % 
    (a) There is no external energy source in the closed system so
    $P_E=0$. Conservation of energy gives $P_C+P_R=0$: the energy
    flowing out of the \C components is dissipated in the \Re components.
    % 
    (b) There is an external energy source in the open system with
    $\mu_e = RT \Ln K_e$. Initially, energy flows out of the the open
    system but, in the steady state, the energy into the system is
    balanced by the energy dissipated in the \Re components and the
    energy flow into the \C components becomes zero.  }
  \label{fig:energy}
\end{figure}

\begin{figure}[htbp]
  \centering
  \Fig{glycolysis_pathway}{0.9}
\caption{The simplified glycolytic pathway as described by Lambeth \& Kushmerick. 
  \textbf{Reactions/enzymes:} GP, glycogen phosphorylase; PGM, phosphoglucomutase; PGI, 
  phosphoglucose isomerase; PFK, phosphofructo kinase; ALD, aldolase; TPI, 
  triose phosphate isomerase; GAPDH, glyceraldehyde 3-phosphate dehydrogenase;
  PGK, phosphoglycerate kinase; PGM, phosphoglycerate mutase; ENO, enolase; 
  PK, pyruvate kinase; LDH, lactate dehydrogenase; CK, creatine kinase; AK, 
  adenylate kinase; ATPase, ATPases. \textbf{Metabolites:} GLY, glycogen;
  P$_{i}$, inorganic phosphate; G1P, glucose-1-phosphate; G6P, glucose-6-phosphate;
  F6P, fructose-6-phosphate; ATP, adenosine triphosphate; ADP, adenosine diphosphate;
  FBP, fructose-1,6-biphosphate; DHAP, dihydroxyacetone phosphate; GAP, 
  glyceraldehyde 3-phosphate; NAD, oxidised nicotinamide adenine dinucleotide; NADH, 
  reduced NAD; 13BPG, 1,3-bisphosphoglycerate; 3PG, 3-phosphoglycerate; 2PG, 
  2-phosphoglycerate; PEP, phosphoenolpyruvic acid; PYR, pyruvate; LAC, lactate;
  PCr, phosphocreatine; Cr, creatine; AMP, adenosine monophosphate. Enzyme 
  commission (EC) numbers are shown.}
  \label{fig:LamKus02}
\end{figure}

\begin{figure}
  \centering
  \SubFig{LamKus02_abg}{\textbf{LamKus02}}{0.45}
  \SubFig{GLY2LAC_abg}{\textbf{GLY2LAC}}{0.45}
  \SubFig{GLY2FBP_abg}{\textbf{GLY2FBP}}{0.9}
  \SubFig{FBP2GAP_abg}{\textbf{FBP2GAP}}{0.45}
  \SubFig{GP_abg}{\textbf{GP}}{0.45}
%%  \SubFig{Rea_abg}{Active reaction}{0.45}
  \caption{Hierarchical Bond Graph Model. (a) As discussed in the
    text, \textbf{LamKus02} represents the top-level model of the
    system in Figure \ref{fig:LamKus02}. (b) \textbf{GLY2LAC}
    represents one of the three submodels in (a). (c)\&(d)
    \textbf{GLY2FBP}\&\textbf{FBP2GAP} represent two of the three
    submodels in (b). (e) The reaction $GP$ has two parallel reactions
    $GPa$ and $GPb$.}
  \label{fig:BG}
\end{figure}
\begin{figure}[htbp]
  \centering
  \SubFig{LamKus02-0_Xeq}{Closed system}{0.45}
  \SubFig{LamKus02-4_Xeq}{Open system}{0.45}
  \caption{Simulation: equilibria. For each reaction, the ratio
    $\Vt_0 = K^{v} \hp X^{v}$  () of the forward to backward reaction
    flows, is plotted. (a) Closed system: each ratio tends to unity:
    the steady state is a thermodynamic equilibrium. (b) Open system:
    some ratios tend to a non-unity value: the steady state is a not a
    thermodynamic equilibrium. }
  \label{fig:equilibria}
\end{figure}

\begin{figure}[htbp]
  \centering
  \SubFig{LamKus02-0_P}{Closed system}{0.45}
  \SubFig{LamKus02-4_P}{Open system}{0.45}
  \caption{Simulation: energy flows. Three energy flows are plotted:
    $P_E$ the external energy flow, $P_R$ the energy dissipated in all
    reactions (including $ATPase$), $P_{ATPase}$ the power consumed by
    the processes represented by the $ATPase$ reaction. (a) There is
    no external energy flow or $ATPase$ reaction flow: $P_R$ tends to
    zero as the energy in the internal species is used up. (b) There
    is external energy flow and $ATPase$ reaction flow: $P_R$ tends to
    $P_E$ as the energy in the internal species is used up. In this
    case, at steady-state about 90\% of the energy is used across
    processes represented by the $ATPase$ reaction.}
  \label{fig:Energy}
\end{figure}

%% REset fig counter to coresp. to paper
\setcounter{figure}{2}
\begin{figure}[htbp]
  \centering
  \SubFig{LamKus02-0_X}{Closed system}{0.45}
  \SubFig{LamKus02-4_X}{Open system}{0.45}
  \caption{[Supplementary Material] Simulated concentrations.  Evolution of the
    concentrations for $ATP$, $ADP$, $P$, $GLY$ and $LAC$ corresponds
    to the situation in Figure 2 of Lambeth\&Kushmerick, 2002 except
    that they use unit initial states. Following an initial transient,
    the species concentrations reach steady-state values for both the
    closed and open systems (cf Figure \ref{fig:sim_ABCAo}).}
  \label{fig:concentrations}
\end{figure}


\begin{figure}[htbp]
  \centering
  \SubFig{LamKus02-0_CM}{Closed system}{0.45}
  \SubFig{LamKus02-4_CM}{Open system}{0.45}
  \caption{[Supplementary Material] Conserved Moieties. The sum of each conserved moiety
    remains constant.}
  \label{fig:moieties}
\end{figure}

\begin{figure}[htbp]
  \centering
  \SubFig{LamKus02-0_FoutATPase}{Closed system}{0.45}
  \SubFig{LamKus02-4_FoutATPase}{Open system}{0.45}
  \caption{[Supplementary Material] Simulated mass flows. Both plots show the input mass flow
    (though \textbf{SS:GLYo}), the output mass flow (though
    \textbf{Re:Fout}) and the ATP flux (though \textbf{Re:ATPase}) (a)
    In the closed system, the flows are zero. (b) In the open system,
    the three flows reach a constant steady state. The final value of
    ATP flow is 5.9mM/min; this is close to to value of 6.1mM/min
    quoted in the ``Moderate exercise'' column of Table 4] of of
    Lambeth\&Kushmerick, 2002}
  \label{fig:mass}
\end{figure}


\end{document}

%%% Local Variables:
%%% End: