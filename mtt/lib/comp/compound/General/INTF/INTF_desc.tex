% Verbal description for system INTF (INTF_desc.tex)
% Generated by MTT on Fri Aug 15 09:53:16 BST 1997.

% %%%%%%%%%%%%%%%%%%%%%%%%%%%%%%%%%%%%%%%%%%%%%%%%%%%%%%%%%%%%%%%
% %% Version control history
% %%%%%%%%%%%%%%%%%%%%%%%%%%%%%%%%%%%%%%%%%%%%%%%%%%%%%%%%%%%%%%%
% %% $Id: INTF_desc.tex,v 1.1 1997/08/24 11:20:18 peterg Exp $
% %% $Log: INTF_desc.tex,v $
% %% Revision 1.1  1997/08/24 11:20:18  peterg
% %% Initial revision
% %%
% %%%%%%%%%%%%%%%%%%%%%%%%%%%%%%%%%%%%%%%%%%%%%%%%%%%%%%%%%%%%%%%

   The acausal bond graph of system \textbf{INTF} is
   displayed in Figure \Ref{INTF_abg} and its label
   file is listed in Section \Ref{sec:INTF_lbl}.
   The subsystems are listed in Section \Ref{sec:INTF_sub}.

\textbf{INTF} is a two-port component where the effort on port [out]
   is the integral of the flow on port [in].
