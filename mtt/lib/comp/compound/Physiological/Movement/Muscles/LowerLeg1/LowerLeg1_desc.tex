% -*-latex-*- Put EMACS into LaTeX-mode
% Verbal description for system LowerLeg1 (LowerLeg1_desc.tex)
% Generated by MTT on Thu Apr 19 10:39:20 BST 2001.

% %%%%%%%%%%%%%%%%%%%%%%%%%%%%%%%%%%%%%%%%%%%%%%%%%%%%%%%%%%%%%%%
% %% Version control history
% %%%%%%%%%%%%%%%%%%%%%%%%%%%%%%%%%%%%%%%%%%%%%%%%%%%%%%%%%%%%%%%
% %% $Id: LowerLeg1_desc.tex,v 1.1 2001/05/01 14:19:07 gawthrop Exp $
% %% $Log: LowerLeg1_desc.tex,v $
% %% Revision 1.1  2001/05/01 14:19:07  gawthrop
% %% Physiological systems used for Teaching Control V at Glasgow
% %%
% %% Revision 1.1  2000/12/28 09:13:38  peterg
% %% Initial revision
% %%
% %%%%%%%%%%%%%%%%%%%%%%%%%%%%%%%%%%%%%%%%%%%%%%%%%%%%%%%%%%%%%%%

\fig{LowerLeg1_pic}
    {LowerLeg1_pic}
    {0.9}
    {Simple lower-leg model}
   
    A schematic of a simple muscle model with skeleton appears in
    Figure \Ref{fig:LowerLeg1_pic} and the acausal bond graph is
    displayed in Figure \Ref{fig:LowerLeg1_abg.ps} and its label file
    is listed in Section \Ref{sec:LowerLeg1_lbl.txt}.  The subsystems
    are listed in Section \Ref{sec:LowerLeg1_sub}.  The model is a
    rough approximation of the lower leg, knee joint and thigh muscles
    with the lower leg horizontal.

    The components are:
    \begin{description}
    \item[F\_1 and F\_2] the force inputs to the upper and lower muscles
    \item[m\_1 and m\_2] the upper and lower muscles
    \item[r\_1 and r\_2] the transformers reflecting the geometry of the
      muscle attachments
    \item[m] load mass
    \item[mg] force due to gravity.
    \end{description}

