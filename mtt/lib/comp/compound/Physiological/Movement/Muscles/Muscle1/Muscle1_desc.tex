% -*-latex-*- Put EMACS into LaTeX-mode
% Verbal description for system Muscle1 (Muscle1_desc.tex)
% Generated by MTT on Thu Apr 19 10:39:29 BST 2001.

% %%%%%%%%%%%%%%%%%%%%%%%%%%%%%%%%%%%%%%%%%%%%%%%%%%%%%%%%%%%%%%%
% %% Version control history
% %%%%%%%%%%%%%%%%%%%%%%%%%%%%%%%%%%%%%%%%%%%%%%%%%%%%%%%%%%%%%%%
% %% $Id: Muscle1_desc.tex,v 1.1 2001/05/01 14:19:07 gawthrop Exp $
% %% $Log: Muscle1_desc.tex,v $
% %% Revision 1.1  2001/05/01 14:19:07  gawthrop
% %% Physiological systems used for Teaching Control V at Glasgow
% %%
% %% Revision 1.1  2000/12/28 09:13:38  peterg
% %% Initial revision
% %%
% %%%%%%%%%%%%%%%%%%%%%%%%%%%%%%%%%%%%%%%%%%%%%%%%%%%%%%%%%%%%%%%

   The acausal bond graph of system \textbf{Muscle1} is
   displayed in Figure \Ref{fig:Muscle1_abg.ps} and its label
   file is listed in Section \Ref{sec:Muscle1_lbl}.
   The subsystems are listed in Section \Ref{sec:Muscle1_sub}.


This model is a highly simplified linear model of a muscle. The two
main components are:
\begin{itemize}
\item a \textbf{CDx} component representing the muscle compliance and
  also providing a measurement of muscle extension and
\item an \textbf{R} component representing mechanical damping in the muscle
\end{itemize}

The component has three ports:
\begin{description}
\item [in] representing the connection of the muscle to the outside world. The
  corresponding force corresponds to the sum of the muscle force and
  the reaction force of connected components.
\item[out] a measurement of the muscle extension.
\item[F] Representing the force component of the muscle model.
\end{description} 

