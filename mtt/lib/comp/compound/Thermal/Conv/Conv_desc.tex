% Verbal description for system Conv (Conv_desc.tex)
% Generated by MTT on Tue Jan 13 18:02:53 GMT 1998.

% %%%%%%%%%%%%%%%%%%%%%%%%%%%%%%%%%%%%%%%%%%%%%%%%%%%%%%%%%%%%%%%
% %% Version control history
% %%%%%%%%%%%%%%%%%%%%%%%%%%%%%%%%%%%%%%%%%%%%%%%%%%%%%%%%%%%%%%%
% %% $Id: Conv_desc.tex,v 1.1 2000/12/28 10:38:59 peterg Exp $
% %% $Log: Conv_desc.tex,v $
% %% Revision 1.1  2000/12/28 10:38:59  peterg
% %% Put under RCS
% %%
% %% Revision 1.2  1998/03/09 10:19:43  peterg
% %% Added note about energy consevation.
% %%
% %% Revision 1.1  1998/03/09 10:13:27  peterg
% %% Initial revision
% %%
% %%%%%%%%%%%%%%%%%%%%%%%%%%%%%%%%%%%%%%%%%%%%%%%%%%%%%%%%%%%%%%%

   The acausal bond graph of system \textbf{Conv} is
   displayed in Figure \Ref{Conv_abg} and its label
   file is listed in Section \Ref{sec:Conv_lbl}.
   The subsystems are listed in Section \Ref{sec:Conv_sub}.

The \textbf{Conv} component represents one way isenentropic flow of 
fluid though a pipe. Externally, it has true energy bonds: $P$/$\dot V$
(Pressure/volume-flow) representing hydraulic energy and $T$/$\dot
S$(Temperature/Entropy-flow) representing convected thermal energy.

Internally, however, the thermal part is represented by a pseudo bond
graph which computes the flow of internal energy $\dot E$ from the
upstream temperature $T_1$ and the mass flow rate $\dot m$ as:
\begin{equation}
  \dot E = c_p T_1 \dot m
\end{equation}
The $AF$ component makes the $FMR$ component use $T_1$ rather than
$T_1-T_2$.

The two \textbf{ES} components provide the conversion from true to
psuedo thermal bonds and vice versa.

%The pipe has an resistance to flow represented by the \textbf{RS}
%component labeled `r' which can be linear or nonlinear. The hydraulic
%energy loss reappears on the thermal bond of this (energy-conserving)
%\textbf{RS} component.

%%% Local Variables: 
%%% mode: latex
%%% TeX-master: t
%%% End: 
