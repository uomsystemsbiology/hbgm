% -*-latex-*- Put EMACS into LaTeX-mode
% Verbal description for system ICV (ICV_desc.tex)
% Generated by MTT on Tue Mar 31 15:58:34 BST 1998.

% %%%%%%%%%%%%%%%%%%%%%%%%%%%%%%%%%%%%%%%%%%%%%%%%%%%%%%%%%%%%%%%
% %% Version control history
% %%%%%%%%%%%%%%%%%%%%%%%%%%%%%%%%%%%%%%%%%%%%%%%%%%%%%%%%%%%%%%%
% %% $Id: ICV_desc.tex,v 1.1 2000/12/28 10:36:01 peterg Exp $
% %% $Log: ICV_desc.tex,v $
% %% Revision 1.1  2000/12/28 10:36:01  peterg
% %% Put under RCS
% %%
% %%%%%%%%%%%%%%%%%%%%%%%%%%%%%%%%%%%%%%%%%%%%%%%%%%%%%%%%%%%%%%%

   The acausal bond graph of system \textbf{ICV} is
   displayed in Figure \Ref{ICV_abg} and its label
   file is listed in Section \Ref{sec:ICV_lbl}.
   The subsystems are listed in Section \Ref{sec:ICV_sub}.
   
   This thermal tank model has been developed to represent an
   inter-volume component (ICV) for a gas turbine. The major
   simplification is that the mass contained in the ICV is assumed
   constant -- this is consistent with using an ideal compressor and an
   ideal turbine with identical mass flows. Energy conservation is
   ensured by using true bonds and \textbf{TF} components.

%%% Local Variables: 
%%% mode: latex
%%% TeX-master: t
%%% End: 
