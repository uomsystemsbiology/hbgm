% -*-latex-*- Put EMACS into LaTeX-mode
% Verbal description for system VMe (VMe_desc.tex)
% Generated by MTT on Thu Mar 19 14:37:43 GMT 1998.

% %%%%%%%%%%%%%%%%%%%%%%%%%%%%%%%%%%%%%%%%%%%%%%%%%%%%%%%%%%%%%%%
% %% Version control history
% %%%%%%%%%%%%%%%%%%%%%%%%%%%%%%%%%%%%%%%%%%%%%%%%%%%%%%%%%%%%%%%
% %% $Id: VMe_desc.tex,v 1.1 2000/12/28 10:41:34 peterg Exp $
% %% $Log: VMe_desc.tex,v $
% %% Revision 1.1  2000/12/28 10:41:34  peterg
% %% Put under RCS
% %%
% %%%%%%%%%%%%%%%%%%%%%%%%%%%%%%%%%%%%%%%%%%%%%%%%%%%%%%%%%%%%%%%

   The acausal bond graph of system \textbf{VMe} is
   displayed in Figure \Ref{VMe_abg} and its label
   file is listed in Section \Ref{sec:VMe_lbl}.
   The subsystems are listed in Section \Ref{sec:VMe_sub}.

The \textbf{VMe} component acts as a density-modulated transformer
converting a $P$/$\dot V$ energy bond to a $Pv$/$\dot m$ bond. It used
the \textbf{Density} component and therfore inherits the properties of
that component -- in particular it takes the same parameter. The four
ports are:

\begin{itemize}
\item [Hy_in]: hydraulic in
\item [Hy_out]: hydraulic out
\item [P]: Pressure
\item [T]: Temperature
\end{itemize}

