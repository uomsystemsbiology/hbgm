\message{ !name(LamKus02_report.tex)}\documentclass[12pt,a4paper]{article}
\title{Bond Graph Model of Glycogenolysis \citep{LamKus02}}
\author{Peter Gawthrop}
%Maths
\usepackage{amsmath,amssymb,amsbsy,amscd,amstext,mathtools,amsxtra}
%%Citations
\usepackage[authoryear,round,sort,longnamesfirst]{natbib}
%%\usepackage[authoryear,round,sort]{natbib}
%%\usepackage[numbers,compress]{natbib}
 \bibliographystyle{unsrtnat}
%%\bibliographystyle{plainnat}

%Chemistry
\newcommand{\reacul}[2]{{\; \xrightleftharpoons[#2]{#1} \;}}
\newcommand{\reacu}[1]{\reacul{#1}{}}
\newcommand{\reac}{\reacu{}}

% Figs
\usepackage{graphicx,subfigure,fancybox,color}

\newcommand{\Fig}[2]{
 \includegraphics[width=#2\linewidth]{Figs/#1.pdf}
}

\newcommand{\SubFig}[3]{
 \subfigure[#2]{
   \includegraphics[width=#3\linewidth]{Figs/#1.pdf}
   \label{subfig:#1}
 }
}

\newcommand{\FIG}[2]{
  \SubFig{LamKus02-#1_ATP}{ATP,ADP,P}{0.45}
  \SubFig{LamKus02-#1_GLY}{GLYf,GLYb,LAC}{0.45}
  \SubFig{LamKus02-#1_Xeq}{$X_{eq}/K_{eq}$}{0.45}
  \SubFig{LamKus02-#1_CM}{Normalised conserved moieties}{0.45}
  \SubFig{LamKus02-#1_extFlow}{External flows}{0.45}
  \SubFig{LamKus02-#1_FoutATPase}{$v_{Fout}$,$v_{ATPase}$}{0.45}
  \caption{Simulation: Mode #1. #2}
  \label{fig:sim_LamKus02-#1}
}
\usepackage{endfloat}
\usepackage[pagebackref]{hyperref}
\hypersetup{colorlinks=true,citecolor=blue,linkcolor=blue,urlcolor=blue}
\usepackage{url,doi}


\begin{document}

\message{ !name(LamKus02_report.tex) !offset(-3) }

\maketitle
\tableofcontents
\section{Introduction}
\label{sec:introduction}
\begin{figure}[htbp]
  \centering
  \Fig{LamKus02_fig1}{0.8}
  \caption{\citet[Figure 1]{LamKus02}}
  \label{fig:LamKus02}
\end{figure}
One purpose of this paper is to introduce Systems Biologists to the
bond graph modelling approach. To this end, a standard metabolic
pathway from the literature is re-implemented and reexamined from the
bond graph point of view.

\citet{LamKus02} consider the glycolytic pathway of Figure \ref{fig:LamKus02} and
provide a detailed thermodynamically consistent model. This model has
been further embellished by  \citet{VinRusPal10} who also provide a
detailed comparison with experimental data. \citet[Chapter 7]{Bea12}
uses this model as an exemplar of the simulation of complex
biochemical cellular systems. \citet{MosAlfMaj12} have used a similar
model to examine the interaction of the AKT pathway on metabolism.


\section{Bond graph model}
\label{sec:bond-graph-model}
\begin{figure}
  \centering
  \Fig{LamKus02_abg}{0.8}
  \caption{Bond graph model}
  \label{fig:bg}
\end{figure}

\begin{figure}
  \centering
  \Fig{GLY2LAC_abg}{0.6}
  \caption{FBP2GAP: $3ADP+3P+GLYf \reac 3ATP+2LAC+GLYr$}
  \label{fig:bg_submodel_1}
\end{figure}

\begin{figure}
  \centering
  \Fig{CrAMP_abg}{0.6}
  \caption{CrAMP: $AMP+2ATP+Cr \reac  3ADP+PCr$}
  \label{fig:bg_submodel_2}
\end{figure}

\begin{figure}
  \centering
  \SubFig{GLY2FBP_abg}{GLY2FBP: $ATP+P+GLYf \reac ADP+FBP+GLYr$}{0.9}
  \SubFig{GAP2LAC_abg}{GAP2LAC: $2ADP+P+GAP \reac 2ATP+LAC$}{0.9}
  \SubFig{FBP2GAP_abg}{FBP2GAP: $FBP \reac 2GAP$}{0.6}
  \caption{Bond-graph submodels}
  \label{fig:bg_submodel_3}
\end{figure}

\citet{LamKus02} present ``A Computational Model for Glycogenolysis
in Skeletal Muscle'' based on the schematic reaction diagram of Figure
\ref{fig:LamKus02}. As a first step in creating a hierarchical model,
the reaction diagram is divided, in terms of function into three conceptual parts:
\begin{enumerate}
\item The reaction chain leading from glycogen ($GLY$) to lactate
  $LAC$. The function of this part is to convert adenosine diphosphate $ADP$ and
  inorganic phosphate $P$ into $ATP$ making use of the energy stored
  in glycogen.
\item The pair of reactions catalysed by creatine kinase $CK$ and
  creatine kinase $ADK$ involving creatin ($Cr$), phosphocreatine
  ($PCr$) and adenosine monophosphate ($AMP$) as well as $ATP$ and
  $ADP$. The function of this part is to act as buffer storage of
  $ATP$.
\item The reaction catalysed by ATPases ($ATPase$) converting $ATP$
  into $ADP$ and $P$ and releasing the stored energy to perform (in this
  case) mechanical work.
\end{enumerate}
Figure \ref{fig:bg} gives a bond-graph representation of this
top-level decomposition were the three parts are the compound bong
graph systems labeled \textbf{GLY2LAC}, \textbf{CrAMP} and 



\subsection{Reactions}
\begin{align}
  \text{ADK}:	&AMP+ATP = 2ADP\\
\text{CK}:	&ATP+Cr = ADP+PCr\\
\text{ALD}:	&FBP = DHAP+GAP\\
\text{TPI}:	&GAP = DHAP\\
\text{ENO}:	&P2G = PEP\\
\text{Fout}:	&LAC = LACo\\
\text{PGM}:	&P3G = P2G\\
\text{GAPDH}:	&P+GAP+NAD = DPG+NADH\\
\text{LDH}:	&NADH+PYR = LAC+NAD\\
\text{PGK}:	&ADP+DPG = ATP+P3G\\
\text{PK}:	&ADP+PEP = ATP+PYR\\
\text{GPa}:	&P = G1P\\
\text{GPb}:	&P = G1P\\
\text{PGI}:	&G6P = F6P\\
\text{PGLM}:	&G1P = G6P\\
\text{PFK}:	&ATP+F6P = ADP+FBP\\
\text{ATPase}:	&ATP = ADP+P\\

\end{align}
% \subsection{ODEs}
% \begin{align}
%   \dot{X}_{ADP} &=  + 2V_{ADK}  + V_{CK}  - V_{PGK}  - V_{PK}  + V_{PFK}  + V_{ATPase} \\
\dot{X}_{AMP} &=  - V_{ADK} \\
\dot{X}_{ATP} &=  - V_{ADK}  - V_{CK}  + V_{PGK}  + V_{PK}  - V_{PFK}  - V_{ATPase} \\
\dot{X}_{Cr} &=  - V_{CK} \\
\dot{X}_{P} &=  - V_{GAPDH}  - V_{GPa}  - V_{GPb}  + V_{ATPase} \\
\dot{X}_{PCr} &=  + V_{CK} \\
\dot{X}_{DHAP} &=  + V_{ALD}  + V_{TPI} \\
\dot{X}_{FBP} &=  - V_{ALD}  + V_{PFK} \\
\dot{X}_{GAP} &=  + V_{ALD}  - V_{TPI}  - V_{GAPDH} \\
\dot{X}_{DPG} &=  + V_{GAPDH}  - V_{PGK} \\
\dot{X}_{LAC} &=  - V_{Fout}  + V_{LDH} \\
\dot{X}_{LACo} &=  + V_{Fout} \\
\dot{X}_{NAD} &=  - V_{GAPDH}  + V_{LDH} \\
\dot{X}_{NADH} &=  + V_{GAPDH}  - V_{LDH} \\
\dot{X}_{P2G} &=  - V_{ENO}  + V_{PGM} \\
\dot{X}_{P3G} &=  - V_{PGM}  + V_{PGK} \\
\dot{X}_{PEP} &=  + V_{ENO}  - V_{PK} \\
\dot{X}_{PYR} &=  - V_{LDH}  + V_{PK} \\
\dot{X}_{F6P} &=  + V_{PGI}  - V_{PFK} \\
\dot{X}_{G1P} &=  + V_{GPa}  + V_{GPb}  - V_{PGLM} \\
\dot{X}_{G6P} &=  - V_{PGI}  + V_{PGLM} \\
\dot{X}_{GLY} &= \\

% \end{align}

% \subsection{Flows}
% \begin{align}
%   V_{ADK} 	&= k(X)(X_{AMP} X_{ATP}  - K_{ADK}X_{ADP}^{2} )\\
V_{CK} 	&= k(X)(X_{ATP} X_{Cr}  - K_{CK}X_{ADP} X_{PCr} )\\
V_{ALD} 	&= k(X)(X_{FBP}  - K_{ALD}X_{DHAP} X_{GAP} )\\
V_{TPI} 	&= k(X)(X_{GAP}  - K_{TPI}X_{DHAP} )\\
V_{ENO} 	&= k(X)(X_{P2G}  - K_{ENO}X_{PEP} )\\
V_{Fout} 	&= k(X)(X_{LAC}  - K_{Fout}X_{LACo} )\\
V_{PGM} 	&= k(X)(X_{P3G}  - K_{PGM}X_{P2G} )\\
V_{GAPDH} 	&= k(X)(X_{P} X_{GAP} X_{NAD}  - K_{GAPDH}X_{DPG} X_{NADH} )\\
V_{LDH} 	&= k(X)(X_{NADH} X_{PYR}  - K_{LDH}X_{LAC} X_{NAD} )\\
V_{PGK} 	&= k(X)(X_{ADP} X_{DPG}  - K_{PGK}X_{ATP} X_{P3G} )\\
V_{PK} 	&= k(X)(X_{ADP} X_{PEP}  - K_{PK}X_{ATP} X_{PYR} )\\
V_{GPa} 	&= k(X)(X_{P}  - K_{GPa}X_{G1P} )\\
V_{GPb} 	&= k(X)(X_{P}  - K_{GPb}X_{G1P} )\\
V_{PGI} 	&= k(X)(X_{G6P}  - K_{PGI}X_{F6P} )\\
V_{PGLM} 	&= k(X)(X_{G1P}  - K_{PGLM}X_{G6P} )\\
V_{PFK} 	&= k(X)(X_{ATP} X_{F6P}  - K_{PFK}X_{ADP} X_{FBP} )\\
V_{ATPase} 	&= k(X)(X_{ATP}  - K_{ATPase}X_{ADP} X_{P} )\\

% \end{align}

\subsection{Subsystem stoichiometry}

\begin{xalignat}{2}
  \label{eq:sub_stoich}
  GLY2FBP&: & ATP+P+GLYf &\reac ADP+FBP+GLYr \\
  FBP2GAP&: & FBP &\reac 2GAP \\
  GAP2LAC&: & 2ADP+P+GAP &\reac 2ATP+LAC \\
  GLY2LAC&: & 3ADP+3P+GLYf &\reac 3ATP+2LAC+GLYr \\
  ATPase&: & ATP &\reac  ADP+P  \\
  CrAMP&: & AMP+2ATP+Cr &\reac  3ADP+PCr 
\end{xalignat}
\begin{table}[htbp]
  \centering
  \begin{tabular}{|l|l|}
    \hline
    Composite reaction&$K_{eq}$\\
    \hline
    GLY2FBP & 760.17\\
    FBP2GAP & 4.940\text{e-6}\\
    GAP2LAC &  7.482\text{e10}\\
    GLY2LAC & 2.102\text{e+19} \\
    CrAMP & 514.9 \\
    ATPase & 2.497\text{e+5} \\
    \hline
  \end{tabular}
  \caption{Equilibrium constants}
  \label{tab:K_eq}
\end{table}

\subsection{Stoichiometric analysis}
\begin{verbatim}
Conserved Moieties (bg0)
(1)	ADP AMP ATP 
(2)	Cr PCr 
(3)	NAD NADH 
(4)	DPG NAD P2G P3G PEP PYR 
(5)	ADP 2*ATP P PCr DHAP 2*FBP GAP 2*DPG P2G P3G PEP F6P G1P G6P 
Pathways (bg)
GPa GPb 
\end{verbatim}
\begin{itemize}
\item 
  CM 1--3 are obvious from the equations.
\item CM (4)
  corresponds to the ``total oxidised'' part of
  \citep[Eqn. (3)]{LamKus02}.
\item CM (5) corresponds to
  \citep[Eqn. (2)]{LamKus02} of which they say ``Correct conservation of
  mass within the model was proven for both open and closed systems by
  calculating the total phosphate using the following equation''.
\item 
  Should there be a CM corresponding to the ``total reduced'' part of
  \citep[Eqn. (3)]{LamKus02} (NADH+GAP+LAC)?
\end{itemize}
The pathway corresponds to the two parallel GP reactions which must
therefore have the same $K_{eq}$. The SBML version collapses these two
reactions in one thus avoiding the issue. \citet{MosAlfMaj12} get this
wrong as follows:

The two reactions labelled \textbf{GPa} and \textbf{GPb} are both of
the form:
\begin{equation}
  \label{eq:GP}
  GLY + Pi \reac GLY + G1P
\end{equation}
Both reactions must therefore have the same equilibrium constant
\begin{equation}\label{eq:Keqb}
  K_{eqb} = K_{eqa}
\end{equation}
Indeed, \citet[Table 1]{LamKus02} have  $K_{eqa} = K_{eqb} = 0.42$.

Although, on p. 17, \citet{MosAlfMaj12} correctly use this value for
$K_{eqa}$, the value on p. 18 incorrectly gives $K_{eqb} = 16.62$
thus \eqref{eq:Keqb} is violated and the model is not
thermodynamically consistent. A glance at \citep[Table 1]{LamKus02}
reveals that  \citet{MosAlfMaj12} have used the equilibrium constant
for Phosphoglucomutase instead of that for Glycogen Phosphorylase B.

\subsection{Simulations}
\label{sec:sim}

\begin{figure}
  \centering
  \FIG{0}{Closed system. ATPase off, Fout off.}
\end{figure}

\begin{figure}
  \centering
  \FIG{1}{Open system. ATPase off, Fout on. GLY free.}
\end{figure}

\begin{figure}
  \centering
  \FIG{2}{Open system. ATPase off, Fout on. GLY fixed.}
\end{figure}

\begin{figure}
  \centering
  \FIG{3}{Open system. ATPase on, Fout on. GLY free.}
\end{figure}

\begin{figure}
  \centering
  \FIG{4}{Open system. ATPase on, Fout on. GLY fixed.}
\end{figure}

\begin{table}[htbp]
  \centering
  \begin{tabular}{|l|l|}
    \hline
    ATPase coefficient & 0.75\\
    \hline
    ATP flux & 6.0318 mM/min\\
    ATP  & 8.04245 mM\\
    ADP  & 0.168962 mM \\
    AMP  & 0.00160618 mM \\
    Pi  & 0.00253925 mM \\
    PCr  & 39.9964 mM \\
    \hline
  \end{tabular}
  \caption{Moderate Exercise}
  \label{tab:ME}
\end{table}
Figures \ref{fig:sim_LamKus02-0}--~\ref{fig:sim_LamKus02-4} contain 6 subplots:
\begin{enumerate}
\item A plot of the concentrations of $ATP$, $ADP$ and $P$  together with
  the quantity $K_{eqcombo}$ (scaled by $10^{-19}$) from Equation (5)
  of \citet{LamKus02}. This is similar to Figure 2 of \citet{LamKus02}.
\item The concentration of the two types of glycogen: $GLY_f$ and
  $GLY_r$ and $LAC$.
\item For each reaction, except $Fout$ and $ATPase$, the quantity
  $\rho_{eq} = X_{eq}/K_{eq}$ where $K_{eq}$
  is the equilibrium constant and $X_{eq} = \exp(N^T \ln X)$ where $N$
  is the stoichimetric matrix. Thus for reaction $ENO$, $X_{eq} =
  \dfrac{X_{PG}}{X_{PEP}}$. At equillibrium, $\rho_{eq}=1$.
\item The normalised conserved moities $\dfrac{1}{GX_0}GX$: this
  should always be unity (sanity check).
\item The internal flows $V_{Fout}$ and $V_{ATPase}$.
\item For each fixed concentration, the external flow required to keep
  the concentration constant. 
\end{enumerate}

The parameters were taken from \citet[Table 3]{LamKus02} with the
exception of $K_{eq}$ for $TPI$ which appears to be inverted (probably
because the reaction is specified in the ``wrong'' direction. This is
verified by the value given by \citet[Table 9.1]{Bea12}.

In each case, the initial conditions corresponded to \citet[Table
3]{LamKus02} together with the two equations gleaned from page 813:
\begin{align}
  \frac{X_{NAD}}{X_{NADH}} &= 1000\\
  X_{Cr} + X_{PCr} &= 40\text{mM}
\end{align}


The following 5 conditions are simulated and shown in Figures
\ref{fig:sim_LamKus02-0}--~\ref{fig:sim_LamKus02-4}:

  \paragraph{Mode 0: Closed system. ATPase off, Fout off.}
  This corresponds to the situation in \citet[Figure 2]{LamKus02}
  except that they use unit initial states. They introduce
  $K_{eqcombo}$ (there equation 5) as the effective equilibrium constant of
  \textbf{GLY2LAC}. They quote a value of $1.5e19$; our parameters
  give $2.1023e19$ (with the modified $K_{eq}$ for TPI). As in their
  Figure 2, this is plotted (scaled by $1e19$) in Figure \ref{fig:sim_LamKus02-0}.
 

  \paragraph{Mode 1: Open system. ATPase off, Fout on. GLY free.}
  LAC drains away and is not replenished.
  \paragraph{Mode 2: Open system. ATPase off, Fout on. GLY fixed.}
  GLYf and GLYr are fixed and the corresponding flows appear in
  subfigure(d). This make little difference to the behaviour in mode 1.
  \paragraph{Mode 3: Open system. ATPase on, Fout on. GLY free.}
  The system reaches a non-equilibrium steady state approximating the
  ``Moderate Exercise'' condition of \citet[Table 4]{LamKus02} -- see
  Table \ref{tab:ME}. In particular, the ATPase flow is about 6mM/min.
  \paragraph{Mode 4: Open system. ATPase on, Fout on. GLY fixed.}
  GLYf and GLYr are fixed and the corresponding flows appear in
  subfigure(d). This make little difference to behaviour in mode 3.
\appendix
\section{Errors and discrepancies in \citet{LamKus02}}
\label{sec:errors-citelamkus02}

\subsection{Glycogen Phosphorylase}
There are a number of apparent issues with the pair of reactions $GPa$
and $GPb$.
\begin{itemize}
\item The $GLY$ appears on both sides of the rate equation. This would
  imply that the rate of change of the concentration $GLY$ is
  zero. But the paper states that ``GLY' = - flux\_GP''. There is an
  argument about this which I don't follow. However, the paper seems
  to set $GLY$ to be a constant so maybe this does not matter.
\item I'm confused about the difference between GPa and the first
  factor of GPb. In particular, the factors multiplying V\_maxf and
  V\_maxr in the numerator also appear identically in the
  denominator. This is true for GPb but not GPa.
  On second thoughts, perhaps the Vmax terms can be adjusted accordingly.
\end{itemize}

\subsection{TPI}
\label{sec:TPI}
The equilibrium constant is given as $K_{eq} =  0.052$. However this
gives the wrong value of $K_{eqcombo}$. Because the reaction is
specified in the ``wrong'' direction, it is assumed that $K_{eq}$
should be the reciprocal of the given value, ie $K_{eq} = 19.23$.
\citet{Bea12} quotes $K_{eq} = 19.87$; so this alteration seems to be
correct.



\subsection{Typographical errors}
\begin{itemize}
\item In \citet[Figure 1]{LamKus02}
  \begin{itemize}
  \item Reaction $PGK$ ATP and ADP are reversed
  \item Reaction $PK$ ATP and ADP are reversed
  \end{itemize}
\end{itemize}

\subsection{Discrepancies in SBML version}
\label{sec:discr-sbml-vers}
The biomodels repository \url{http://www.ebi.ac.uk/biomodels-main/}
contains an SBML version (MODEL6623617994)
\url{http://www.ebi.ac.uk/biomodels-main/MODEL6623617994} of the model
of \citet{LamKus02}. There are a number of discrepancies:
\begin{itemize}
\item Reaction $FBP$ is renamed as $FDP$
\item There is an extra reaction $fout$ and corresponding state
  $LACo$. This presumably corresponds to the ``lactate efflux''
  referred to in the paper. This has been added to the BG model as
  part of \textbf{GAP2LAC}.
\item The parameters are different.
\end{itemize}

\section{Comparison of BG and SBML}
With the following changes the stoichiometric matrices are identical:
\begin{enumerate}
\item (BG) The two states $GLYf$ and $GLYr$ are removed.
\item (BG) The two reactions $GPa$ and $GPb$ are collapsed into one and
  renamed $GP$.
\item (SBML) The states and reactions are reordered to correspond to
  the ordering of the BG model.
\end{enumerate}

\bibliography{common}
\end{document}

%%% Local Variables: 
%%% mode: latex
%%% TeX-master: t
%%% End: 

\message{ !name(LamKus02_report.tex) !offset(-398) }
